\documentclass{article}
\usepackage{amsmath, amsthm, amssymb}
\usepackage{tikz}
\usetikzlibrary{positioning,automata}
\usepackage{array}
\usepackage{mathtools}
\usepackage{graphicx}

\begin{document}
\section*{\huge Homework Sheet 4}
\begin{flushright}
   \textbf{Author: Abdullah Oğuz Topçuoğlu \& Yousef Mostafa Farouk}
\end{flushright}

% Task 1 (4 points) Let Σ = {0, 1}. For x = xn−1xn−2 · · · x1x0, let ⟨x⟩2 =
% P
% 0≤i<n xi2
% i be the
% number denoted by string x in the binary system.
% Develop a regular expression that denotes exactly all binary strings u with ⟨u⟩2 mod 3 = 1.
% Give an argument, why this expression indeed denotes the desired language.
% Hint: A possible way of proceeding could be to first construct a DFA and then the regular
% expression.
% Task 2 (8 points) Classify each of the following languages with respect to being regular, contextfree but not regular, or non-context-free. Prove your answers.
% 1. L1 = {a
% n
% b
% mc
% p
% |n + 2m > 3p}
% 2. L2 = {w ∈ {a, b}
% ∗
% |#a(w) ≤ #b(w)/2}
% 3. L3 = {w ∈ {a, b}
% ∗
% |#a(w) mod 5 ≥ #b(w) mod 3}
% 4. L4 = {w ∈ {a, b}
% ∗
% |#a(w) mod #b(w) = 0}
% Task 3 (4 points) For two strings x = x1 . . . xn and y = y1 . . . ym, we define the language
% merge(x, y) as follows:
% merge(x, ε) = {x}
% merge(ε, y) = {y}
% merge(x, y) = {x1} · merge(x2 . . . xn, y) ∪ {y1} · merge(x, y2 . . . ym)
% In other words, this language contains all possible words that arise from interleaving the
% symbols of x and y while maintaining the relative order of the symbols in each string.
% For two languages L and L
% ′ we define
% merge(L, L′
% ) = [
% {merge(x, x′
% )|x ∈ L, x′ ∈ L
% ′
% } .
% 1. Determine merge(abc, bd) .
% 2. Are the regular languages closed under merge() ?
% 3. Are the context free languages closed under merge() ?
% 4. What can you say about merge(L, R) for the case where L is context-free and R is
% regular?
% Prove your answers

\section*{Task 2}
\subsection*{(i)}
\begin{align*}
L_1 &= \{a^n b^m c^p \mid n + 2m > 3p\} \\
\end{align*}

We can create a PDA that accepts this langauge.

% 2.1 pda.jpeg here
\begin{figure}[h!]
    \centering
    \includegraphics[width=1\textwidth]{2.1 pda.jpeg}
    \caption{PDA for $L_1$}
    \label{fig:pda}
\end{figure}

Where input alphabet, states and stack alphabet are
\begin{align*}
    \Sigma &= \{a, b, c\} \\
    Q &= \{A, B, C, F\} \\
    \Gamma &= \{X, \text{€}\} \\
\end{align*}
\(L_1\) is context free since there is a PDA that accepts it.

\subsection*{(ii)}
\begin{align*}
L_2 &= \{w \in \{a, b\}^* \mid \#_a(w) \leq \#_b(w)/2\} \\
\end{align*}

\subsection*{(iii)}
\begin{align*}
L_3 &= \{w \in \{a, b\}^* \mid \#_a(w) \mod 5 \geq \#_b(w) \mod 3\} \\
\end{align*}

\(L_3\) is regular since we can create a DFA that accepts it. When we look at the condition there is a finite number of combinations of \(\#_a(w) \mod 5\) and \(\#_b(w) \mod 3\). We can create states for each combination and transition between them based on the input symbols.
\\
We can define the DFA like this
\begin{align*}
    Q &= \{(i, j) \mid i \in \{0, 1, 2, 3, 4\}, j \in \{0, 1, 2\}\} \\
    \Sigma &= \{a, b\} \\
    \delta((i, j), a) &= ((i + 1) \mod 5, j) \\
    \delta((i, j), b) &= (i, (j + 1) \mod 3) \\
    q_0 &= (0, 0) \\
    F &= \{(i, j) \mid i \geq j\} \\
\end{align*}

\subsection*{(iv)}
\begin{align*}
L_4 &= \{w \in \{a, b\}^* \mid \#_a(w) \mod \#_b(w) = 0\} \\
\end{align*}

\(L_4\) is not context free. We gonna play the pumping game to prove it. The pumping game is like this:
\begin{itemize}
    \item Adversary chooses an \(N\).
    \item We choose \(z \in L, \quad |z| \geq N\)
    \item Adversary chooses \(z = uvwxy, \quad |vx| \geq 1, \quad vwx \leq N\)
    \item We choose an \(i\) such that \(uv^iwx^iy \notin L\)
\end{itemize}

(these game rules are copy paste from the lecture) \\
Lets start the game \\
Adversary chooses an \(N\). \\
We choose \(z = a^N b^N \in L_4\) \\
Adversary chooses \(z = uvwxy, \quad |vx| \geq 1, \quad vwx \leq N\) \\



\end{document}