\documentclass{article}
\usepackage{amsmath, amsthm, amssymb}
\usepackage{tikz}
\usetikzlibrary{positioning,automata}
\usepackage{array}
\usepackage{mathtools}
\usepackage{graphicx}

\begin{document}
\section*{\huge Homework Sheet 4}
\begin{flushright}
   \textbf{Author: Abdullah Oğuz Topçuoğlu \& Yousef Mostafa Farouk}
\end{flushright}

% Task 1 (4 points) Let Σ = {0, 1}. For x = xn−1xn−2 · · · x1x0, let ⟨x⟩2 =
% P
% 0≤i<n xi2
% i be the
% number denoted by string x in the binary system.
% Develop a regular expression that denotes exactly all binary strings u with ⟨u⟩2 mod 3 = 1.
% Give an argument, why this expression indeed denotes the desired language.
% Hint: A possible way of proceeding could be to first construct a DFA and then the regular
% expression.
% Task 2 (8 points) Classify each of the following languages with respect to being regular, contextfree but not regular, or non-context-free. Prove your answers.
% 1. L1 = {a
% n
% b
% mc
% p
% |n + 2m > 3p}
% 2. L2 = {w ∈ {a, b}
% ∗
% |#a(w) ≤ #b(w)/2}
% 3. L3 = {w ∈ {a, b}
% ∗
% |#a(w) mod 5 ≥ #b(w) mod 3}
% 4. L4 = {w ∈ {a, b}
% ∗
% |#a(w) mod #b(w) = 0}
% Task 3 (4 points) For two strings x = x1 . . . xn and y = y1 . . . ym, we define the language
% merge(x, y) as follows:
% merge(x, ε) = {x}
% merge(ε, y) = {y}
% merge(x, y) = {x1} · merge(x2 . . . xn, y) ∪ {y1} · merge(x, y2 . . . ym)
% In other words, this language contains all possible words that arise from interleaving the
% symbols of x and y while maintaining the relative order of the symbols in each string.
% For two languages L and L
% ′ we define
% merge(L, L′
% ) = [
% {merge(x, x′
% )|x ∈ L, x′ ∈ L
% ′
% } .
% 1. Determine merge(abc, bd) .
% 2. Are the regular languages closed under merge() ?
% 3. Are the context free languages closed under merge() ?
% 4. What can you say about merge(L, R) for the case where L is context-free and R is
% regular?
% Prove your answers

\section*{Task 1}
Lets first construct the DFA like this: \\
We know that if the sum of even indexed digits minus the sum of odd indexed digits is zero modulo three then the number is zero modulo three. And thats what i will use to
construct the DFA. \\
The machine needs to know what modulo of three we are currently at and also the index of the current digit. \(3 \cdot 2 = 6\) states the machine needs.

\begin{figure}[h!]
    \centering
    \includegraphics[width=1\textwidth]{1dfa.jpeg}
    \caption{DFA}
    \label{fig:my_label}
\end{figure}

Accepting states are "even 1" and "odd 1" and the starting state is "even 0" \\
The regular expression would be:
(0*1(01*0)*1)*0*1(01*0)* \\
This regular expression works because it is generated from the DFA above and they accept the same language. \\
\end{document}