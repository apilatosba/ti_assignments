\documentclass{article}
\usepackage{amsmath, amsthm, amssymb}
\usepackage{tikz}
\usepackage{array}
\usepackage{mathtools}

\begin{document}
\section*{\huge Homework Sheet 1}
\begin{flushright}
   \textbf{Author: Abdullah Oğuz Topçuoğlu}
\end{flushright}

% Introduction to Theoretical CS Winter Term 2025/2026
% Problem Set 1 Due Oct. 24th 8am
% Remember that you can only form groups until October 21rd, 23:59 in the CMS. Submissions
% will be possible no earlier than October 22th. For more information on the exercise sheet modalities,
% see the main page of the course on the CMS.
% Task 1 (4 points)
% 1. (1 point) Let us define a relation R on a set A to be antisymmetric if for all a, a0 ∈ A
% we have (a, a0
% ) ∈ R ⇐⇒ (a
% 0
% , a) ∈/ R. Why is this not a particularly sensible definition?
% Propose an alternative, more sensible definition, that fits the word „antisymmetric“
% better.
% 2. (3 points) Given functions f : A → B and g : B → C we define their composition
% (g ◦ f) as a function of type A → C where (g ◦ f)(a) = g(f(a)).
% Consider the function properties injective, surjective, bijective.
% Investigate whether the following statements are true. If yes, prove them, if no, give a
% small counterexample:
% (a) f, g bijective =⇒ g ◦ f bijective.
% (b) f injective, g surjective =⇒ g ◦ f injective.
% (c) f injective, g surjective =⇒ g ◦ f surjective.
% Task 2 (4 points)
% 1. (2 points) Let A and B be finite sets. Prove that |BA| = |B|
% |A|
% .
% 2. (2 points) Let A be a (not necessarily finite) set, a ∈ A and let k ∈ IN, k ≥ 1.
% Construct a bijection between A
% k
% 
% and A\{a}
% k
% 
% ∪
% A\{a}
% k−1
% 
% . Conclude the well-known formula
% for the binomial coefficient
% 
% n
% k
% 
% =
% 
% n − 1
% k
% 
% +
% 
% n − 1
% k − 1
% 
% .
% Note that 0
% 0 =
%
% 0
% 0
% 
% = 1. You may use that
% 
% 
% 
% A
% k
% 
% 
% 
%  =
%
% |A|
% k
% 
% for finite A.
% Task 3 (4 points)
% Assume that set B contains more than one element and let A be an arbitrary set. Show that
% in this case it is definitely not the case that |BA| ≤ |A|. In other words, BA has strictly larger
% cardinality than A, i.e. there is no surjective mapping from A to BA.
% Task 4 (4 points)
% Let k be some integer in IN. We call a function f : IN → {0, . . . , k} monotonic if for every
% n ∈ IN we have f(n) ≤ f(n + 1).
% Prove that the set of monotonic functions from IN to {0, . . . , k} is countable.

\section*{Problem 1}

\subsection*{(1.)}
The given definition of antisymmetry is not sensible because it implies that for any elements \( a, a' \in A \), if \( (a, a') \in R \), then \( (a', a) \notin R \).
When I hear "antisymmetric" i think of something that is the opposite of symmetric and symmetric means if \( (a, a') \in R \) then \( (a', a) \in R \). And opposite of that would be
if \( (a, a') \in R \), then \( (a', a) \notin R \) as in the question but with a difference that it only applies when \( a \neq a' \). Because otherwise in the
current definition if we plug in \( a = a' \), we get \( (a, a) \in R \iff (a, a) \notin R \) which is a contradiction.


\end{document}