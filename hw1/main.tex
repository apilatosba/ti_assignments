\documentclass{article}
\usepackage{amsmath, amsthm, amssymb}
\usepackage{tikz}
\usepackage{array}
\usepackage{mathtools}

\begin{document}
\section*{\huge Homework Sheet 1}
\begin{flushright}
   \textbf{Author: Abdullah Oğuz Topçuoğlu}
\end{flushright}

% Introduction to Theoretical CS Winter Term 2025/2026
% Problem Set 1 Due Oct. 24th 8am
% Remember that you can only form groups until October 21rd, 23:59 in the CMS. Submissions
% will be possible no earlier than October 22th. For more information on the exercise sheet modalities,
% see the main page of the course on the CMS.
% Task 1 (4 points)
% 1. (1 point) Let us define a relation R on a set A to be antisymmetric if for all a, a0 ∈ A
% we have (a, a0
% ) ∈ R ⇐⇒ (a
% 0
% , a) ∈/ R. Why is this not a particularly sensible definition?
% Propose an alternative, more sensible definition, that fits the word „antisymmetric“
% better.
% 2. (3 points) Given functions f : A → B and g : B → C we define their composition
% (g ◦ f) as a function of type A → C where (g ◦ f)(a) = g(f(a)).
% Consider the function properties injective, surjective, bijective.
% Investigate whether the following statements are true. If yes, prove them, if no, give a
% small counterexample:
% (a) f, g bijective =⇒ g ◦ f bijective.
% (b) f injective, g surjective =⇒ g ◦ f injective.
% (c) f injective, g surjective =⇒ g ◦ f surjective.
% Task 2 (4 points)
% 1. (2 points) Let A and B be finite sets. Prove that |BA| = |B|
% |A|
% .
% 2. (2 points) Let A be a (not necessarily finite) set, a ∈ A and let k ∈ IN, k ≥ 1.
% Construct a bijection between A
% k
% 
% and A\{a}
% k
% 
% ∪
% A\{a}
% k−1
% 
% . Conclude the well-known formula
% for the binomial coefficient
% 
% n
% k
% 
% =
% 
% n − 1
% k
% 
% +
% 
% n − 1
% k − 1
% 
% .
% Note that 0
% 0 =
%
% 0
% 0
% 
% = 1. You may use that
% 
% 
% 
% A
% k
% 
% 
% 
%  =
%
% |A|
% k
% 
% for finite A.
% Task 3 (4 points)
% Assume that set B contains more than one element and let A be an arbitrary set. Show that
% in this case it is definitely not the case that |BA| ≤ |A|. In other words, BA has strictly larger
% cardinality than A, i.e. there is no surjective mapping from A to BA.
% Task 4 (4 points)
% Let k be some integer in IN. We call a function f : IN → {0, . . . , k} monotonic if for every
% n ∈ IN we have f(n) ≤ f(n + 1).
% Prove that the set of monotonic functions from IN to {0, . . . , k} is countable.

\section*{Problem 1}

\subsection*{(1.)}
The given definition of antisymmetry is not sensible because it implies that for any elements \( a, a' \in A \), if \( (a, a') \in R \), then \( (a', a) \notin R \).
When I hear "antisymmetric" i think of something that is the opposite of symmetric and symmetric means if \( (a, a') \in R \) then \( (a', a) \in R \). And opposite of that would be
if \( (a, a') \in R \), then \( (a', a) \notin R \) as in the question but with a difference that it only applies when \( a \neq a' \). Because otherwise in the
current definition if we plug in \( a = a' \), we get \( (a, a) \in R \iff (a, a) \notin R \) which is a contradiction.

\subsection*{(2.)}
\textbf{(a)} True.\\
If \( f: A \to B \) and \( g: B \to C \) are bijective functions.\\
To prove that \( g \circ f \) is bijective, we need to show that it is both injective and surjective.\\
Since \(f\) and \(g\) are injective their composition is also injective. And since \(f\) and \(g\) are surjective their composition is also surjective.\\
Therefore, \( g \circ f \) is bijective.\\
\\
\textbf{(b)} False.\\
Let \( A = \{1, 2\}, B = \{1, 2\}, C = \{1\} \).\\
Define \( f: A \to B \) by \( f(1) = 1 \), \( f(2) = 2 \) (which is injective) and \( g: B \to C \) by \( g(1) = 1, g(2) = 1 \) (which is surjective).\\
Then, \( g \circ f(1) = g(f(1)) = g(1) = 1 \).\\
Also, \( g \circ f(2) = g(f(2)) = g(2) = 1 \).\\
Therefore \( g \circ f \) is not injective since \( g \circ f(1) = g \circ f(2) \).\\
\\
\textbf{(c)} False.\\
Consider this example: Let \( A = \{1\}, B = \{1, 2\}, C = \{1, 2\} \).\\
Define \( f: A \to B \) by \( f(1) = 1 \) (which is injective) and \( g: B \to C \) by \( g(1) = 1, g(2) = 2 \) (which is surjective).\\
In this configuration there is no element in \( A \) that maps to \( 2 \) in \( C \) through \( g \circ f \).\\
Thus, \( g \circ f \) is not surjective.\\

\section*{Problem 2}
\subsection*{(1.)}
\(B^A\) is set of all functions from \(A\) to \(B\). And a function is a relation on \(A \times B\) such that for every \(a \in A\) there is exactly one \(b \in B\) such that \((a, b)\) is in the relation.
So \(|B^A|\) is just how many different ways to find such a relation. For every element in \(A\) we have \(|B|\) choices to map it to an element in \(B\).
Which is \(|B| \times |B| \times ... \times |B|\) ( \(|A|\) times) = \(|B|^{|A|}\). \\
Thats what we wanted to show.

\subsection*{(2.)}
Fix a set A and \(a \in A\) and \(k \in \mathbb{N}\). We need to find a bijection between:
\begin{align*}
   \binom{A}{k} \leftrightarrow \binom{A \setminus \{a\}}{k} \cup \binom{A \setminus \{a\}}{k-1}
\end{align*}

Define a function \( f: \binom{A}{k} \to \binom{A \setminus \{a\}}{k} \cup \binom{A \setminus \{a\}}{k-1} \) as follows:
\begin{align*}
   f(S) = 
   \begin{cases} 
      S & \text{if } a \notin S \\
      S \setminus \{a\} & \text{if } a \in S 
   \end{cases}
\end{align*}

In the first case \(S \in \binom{A \setminus \{a\}}{k}\) and in the second case \(S \setminus \{a\} \in \binom{A \setminus \{a\}}{k-1}\). Now we need to show that this function
is bijective by showing that it is both injective and surjective.\\
\textbf{Injective:} Assume \( f(S_1) = f(S_2) \) for some \( S_1, S_2 \in \binom{A}{k} \). We need to show that \( S_1 = S_2 \).\\
If \( a \notin S_1 \) and \( a \notin S_2 \), then \( f(S_1) = S_1 \) and \( f(S_2) = S_2 \). Thus, \( S_1 = S_2 \).\\
If \( a \in S_1 \) and \( a \in S_2 \), then \( f(S_1) = S_1 \setminus \{a\} \) and \( f(S_2) = S_2 \setminus \{a\} \). Thus, \( S_1 \setminus \{a\} = S_2 \setminus \{a\} \) which implies \( S_1 = S_2 \).\\
If \( a \in S_1 \) and \( a \notin S_2 \), then \( f(S_1) = S_1 \setminus \{a\} \) and \( f(S_2) = S_2 \). This leads to a contradiction since \( S_1 \setminus \{a\} \) has size \( k-1 \) while \( S_2 \) has size \( k \).\\
Similarly, if \( a \notin S_1 \) and \( a \in S_2 \), we reach a contradiction.\\
Thus, \( f \) is injective.\\
\\
\textbf{Surjective:} Let \( T \in \binom{A \setminus \{a\}}{k} \cup \binom{A \setminus \{a\}}{k-1} \). We need to find \( S \in \binom{A}{k} \) such that \( f(S) = T \).\\
If \( T \in \binom{A \setminus \{a\}}{k} \), then let \( S = T \). Then, \( f(S) = S = T \).\\
If \( T \in \binom{A \setminus \{a\}}{k-1} \), then let \( S = T \cup \{a\} \). Then, \( f(S) = S \setminus \{a\} = T \).\\
Thus, \( f \) is surjective.\\
Since \( f \) is both injective and surjective, it is bijective.\\
Thats what we wanted to show.

\section*{Problem 3}
We want to show that \(|B^A| > |A|\).\\
If \(A\) and \(B\) are finite sets then we can use the coclusion from Problem 2.1 that \(|B^A| = |B|^{|A|}\). And since \(|B| > 1\) we have \(|B|^{|A|} > |A|\).\\
In other cases, \\
Assume for the sake of contradiction that there exists a surjective function \( f: A \to B^A \). This means that for every function \( g: A \to B \), there exists an element \( a \in A \) such that \( f(a) = g \).\\
Now, we can construct a function \( h: A \to B \) such that for each \( a \in A \), \( h(a) \) is different from \( f(a)(a) \). This is possible since \( B \) has more than one element.\\
However, by construction, \( h \) cannot be equal to \( f(a) \) for any \( a \in A \), which is a contradiction.\\
Therefore, there is no surjective mapping from \( A \to B^A \), and thus \(|B^A| > |A|\).


\end{document}