\documentclass{article}
\usepackage{amsmath, amsthm, amssymb}
\usepackage{tikz}
\usetikzlibrary{positioning,automata}
\usepackage{array}
\usepackage{mathtools}
\usepackage{graphicx}

\begin{document}
\section*{\huge Homework Sheet 7}
\begin{flushright}
   \textbf{Author: Abdullah Oğuz Topçuoğlu \& Yousef Mostafa Farouk}
\end{flushright}

% Task 3 (5 points) Determine the truth status of each of the following implications. L always stands
% for some language over some finite alphabet. Briefly justify your answers.
% 1. L r.e. =⇒ L r.e.
% 2. L decidable =⇒ L decidable
% 3. L decidable and L not r.e. =⇒ L regular
% 4. L ⪯ L′ and L r.e. =⇒ L′ r.e.
% 5. L ⪯ L′ and L′ r.e. =⇒ L r.e.
% 6. L ⪯ L′ and L not r.e. =⇒ L′ not r.e.
% 7. L regular =⇒ L decidable
% 8. L context-free =⇒ L r.e.
% 9. L context-free =⇒ L decidable
% 10. L = {w ∈ {0, 1}∗ |L(Mw) is context-free} =⇒ L is decidable

\section*{Task 3}
\subsection*{(1) \(L\) r.e. \(\Rightarrow\) \(\overline{L}\) r.e.}
\textbf{FALSE.}
\\
\\
Consider the language SAM. In the lecture we saw that SAM is r.e. but its complement \(\overline{SAM}\) is not r.e.
\\
\\

\subsection*{(2) \(L\) decidable \(\Rightarrow\) \(\overline{L}\) decidable}
\textbf{TRUE.}
\\
\\
Using the definition of decidability, L is decidable means that there is a TM that accepts L and there is another TM that accepts \(\overline{L}\). \\
\(\overline{L}\) is decidable means the same thing but in reverse order which doesnt matter.

\subsection*{(3) \(L\) decidable and \(\overline{L}\) not r.e. \(\Rightarrow\) \(L\) regular}
\textbf{TRUE.}
\\
\\
From false we can imply anything. The premise is false because from (2) we know that if L is deciadable then is \(\overline{L}\). \(\overline{L}\) being decidable implies
\(\overline{L}\) is r.e..
\\
\\

\subsection*{(4) \(L \preceq L'\) and \(L\) r.e. \(\Rightarrow\) \(L'\) r.e.}
\textbf{FALSE.}
\\
\\
\textbf{Counterexample:} Let \(L = \emptyset\) (the empty language) and \(L'\) a language that is not r.e.
\begin{itemize}
   \item \(L = \emptyset\) is r.e. (trivially, the TM that rejects everything recognizes it).
   \item \(L \preceq L'\): The reduction \(f(x)\) goes from \(L\) to \(L'\) but since \(L\) is empty, the condition \(x \in L \Leftrightarrow f(x) \in L'\) is trivially true.
\end{itemize}
So in this example the premises hold, but the conclusion does not.

\subsection*{(5) \(L \preceq L'\) and \(L'\) r.e. \(\Rightarrow\) \(L\) r.e.}
\textbf{TRUE.}
\\
\\
\textbf{Proof:} Since \(L \preceq L'\), there exists a computable function \(f\) such that \(x \in L \Leftrightarrow f(x) \in L'\).
\\
\\
Since \(L'\) is r.e., there is a TM \(M'\) that recognizes \(L'\).
\\
\\
To recognize \(L\): on input \(x\), compute \(f(x)\) and then run \(M'\) on \(f(x)\). Accept if \(M'\) accepts.
\\
\\
This machine accepts \(x\) iff \(f(x) \in L'\) iff \(x \in L\). So \(L\) is r.e.

\subsection*{(6) \(L \preceq L'\) and \(L\) not r.e. \(\Rightarrow\) \(L'\) not r.e.}
\textbf{TRUE.}
\\
\\
If \(L'\) were r.e. then by (5) \(L\) would be r.e., contradiction. So \(L'\) cannot be r.e.

\subsection*{(7) \(L\) regular \(\Rightarrow\) \(L\) decidable}
\textbf{TRUE.}
\\
\\
If L is regular then its complement \(\overline{L}\) is also regular. If a DFA accepts L then there is a TM that accepts L by simulating the DFA. \\
So there is a TM that accepts L and there is another TM that accepts \(\overline{L}\). Meaning that L is decidable.

\subsection*{(8) \(L\) context free \(\Rightarrow\) \(L\) r.e.}
\textbf{TRUE.}
\\
\\
If L is context free then there is a NPDA that accepts it. Since TMs are more powerful than NPDAs, there is a TM that accepts L which means that L is r.e..

\subsection*{(9) \(L\) context free \(\Rightarrow\) \(L\) decidable}
\textbf{TRUE.}
\\
\\
In the lecture we saw that there is an algorithm that decides whether a string can be derived from a context free grammar. \\
We can construct a TM that simulates this algorithm to decide L. So L is decidable.

\subsection*{(10) \(L = \{w \in \{0,1\}^* \mid L(M_w) \text{ is context free}\} \Rightarrow L\) is decidable}
\textbf{FALSE.}
\\
\\
From Rice's theroem we know that a language with the following form is undecidable:
\begin{align*}
   L = \{ w \mid L(M_w) \text{ satisfies a property} \}
\end{align*}
In this case the property is being context free but that doesnt matter. L is undecidable.

\end{document}
